Pro tvorbu materiálů pro podporu studia, jako jsou skripta, handouty ke cvičením a různé další studijní materiály, lze
využít mnoha různých formátů, které mají každý své výhody a nevýhody. Zejména techničtější předměty v materiálech
typicky zahrnují množství různých matematických výrazů, diagramů, grafů nebo výpisů kódu. Dlouhodobě využívaným
standardem pro tvorbu těchto typů dokumentů je \LaTeX{}, díky jeho rozsáhlým typografickým možnostem pro velice přesnou
sazbu. Tento formát však má strmější učící křivku oproti například Word dokumentům. Dalším z problémů tohoto formátu je
překlad do HTML, jenž existuje pouze v nepříliš kvalitní podobě díky různým externím nástrojům.

Na Fakultě informačních technologií Českého vysokého učení technického v Praze proto vznikl z iniciativy Ing. Tomáše
Kalvody, Ph.D. z Katedry aplikované matematiky zcela nový formát nazývaný WooWoo, již v současnosti využívaný pro tvorbu
studijních textů pro více předmětů.

Tento formát se snaží vyřešit problém oddělení samotného obsahu dokumentů od informací o způsobu jejich zobrazení a
spolu s tím problém kvalitního víceformátového (HTML a PDF; v budoucnu plánován také EPUB) výstupu z jednoho zdroje.
Zatím však pro něj neexistuje podpora v žádném editoru a autor dokumentů je tak odkázán na psaní strohého zdroje, kde
jedinou zpětnou vazbou je výsledný výstup, jehož vygenerování navíc trvá nezanedbatelnou dobu (v případě materiálů pro
výše zmíněné předměty se jedná i o více než 10 minut).

Cílem této práce je seznámit se s tímto nově vzniklým formátem, prozkoumat možnosti moderních editorů či vývojových
prostředí, zejména co se týče jejich rozšiřitelnosti a navrhnout, implementovat a otestovat rozšíření vybraného editoru,
jehož účelem má být právě poskytnutí chybějící okamžité zpětné vazby při tvorbě WooWoo dokumentů a tím zjednodušení
práce s nimi. Zejména jde pak o přehlednou prezentaci logické struktury dokumentů, zobrazování matematických výrazů a
různých grafických objektů (TikZ diagramy), navigaci mezi různými částmi dokumentu a vyhledávání pomocí značek objektů.
Při plnění těchto cílů bude využito stylu založeného na existujících šablonách.
