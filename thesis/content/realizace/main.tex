Pro realizaci rozšíření Atomu byl vybrán jazyk TypeScript. TypeScript je dle \cite{ts-docs} open-source nadstavba jazyka
JavaScript, která jej rozšiřuje o statickou typovou kontrolu. Díky kompilaci do jazyka JavaScript je pak výstup možné
použít v Atomu.

Kód rozšíření je doplněn o dokumentační komentáře. Dále byl využit nástroj Prettier pro formátování kódu a nástroj
ESLint pro statickou analýzu.

\section{Zvýrazňování syntaxe}
Funkční požadavek \textbf{\ref{F1}} byl na základě provedené rešerše \ref{zvyraznovani-syntaxe} realizován vytvořením
TextMate gramatiky. Tato gramatika je tvořena \textit{grammar} souborem ve složce \mintinline{text}{grammars}. Dynamické
tvorby gramatiky za pomoci API prozatím nebylo využito, její použití zůstává možným vylepšením do budoucna. Zvýrazňování
syntaxe WooWoo zdroje studijního textu k BI-PKM lze vidět na obrázku \ref{zvyraznovani-syntaxe-pkm}.

\begin{figure}\centering
    \includegraphics[width=1.0\textwidth]{content/realizace/zvýrazňování-syntaxe-pkm}
 	\caption[Zvýrazňování WooWoo syntaxe]{Zvýrazňování syntaxe WooWoo zdroje studijního textu k BI-PKM \cite{pkm}}
    \label{zvyraznovani-syntaxe-pkm}
\end{figure}

Podporována je také aktuální zkrácená forma vnitřních prostředí se zavináčem i starší zkrácená forma s tečkou. Jediným
prvkem WooWoo formátu, který není podporován, je direktiva \mintinline{text}{.include}.

Pro zajištění podpory napříč různými populárními tématy syntaxe byla použita pouze obvykle užívaná jména \textit
{scopes}, jejichž seznam uvádí \cite{textmate-grammars}.

\begin{sloppypar}
Vzhledem k omezeným vyjadřovacím schopnostem TextMate gramatik (které jsou založeny na rozšířených regulárních výrazech)
jsou některé konstrukty zvýrazňovány i tehdy, když nejsou zcela syntakticky správné. Například zvýrazňování syntaxe
verbózní formy vnitřních prostředí je řešeno zvýrazňováním ukončující části prostředí (horní dvojitá uvozovka
následovaná tečkou a typem prostředí). Na obrázku \ref{zvyraznovani-syntaxe-chyba-1} je ilustrován případ, kdy je
zvýrazněna syntaxe nevalidního vnitřního prostředí. Další chyby při zvýrazňování syntaxe jsou způsobeny nemožností
určit kontext. Jedna z takových chyb je ilustrována na obrázku \ref{zvyraznovani-syntaxe-chyba-2}, kde je chybně
zvýrazněno \uv{vnitřní prostředí} v křehkém vnějším prostředí.
\end{sloppypar}

\begin{figure}\centering
    \includegraphics[width=0.35\textwidth]{content/realizace/zvýrazňování-syntaxe-chyba-1}
 	\caption[Chyba zvýrazňování syntaxe vnitřních prostředí]{Chyba zvýrazňování syntaxe WooWoo vnitřních prostředí}
    \label{zvyraznovani-syntaxe-chyba-1}
\end{figure}

\begin{figure}\centering
    \includegraphics[width=0.35\textwidth]{content/realizace/zvýrazňování-syntaxe-chyba-2}
 	\caption[Chybějící kontext při zvýrazňování WooWoo syntaxe]{Chybějící kontext při zvýrazňování WooWoo syntaxe}
    \label{zvyraznovani-syntaxe-chyba-2}
\end{figure}


\section{Parsování}
Pro parsování dokumentu do AST byl vytvořen vlastní parser. Tvorba parseru se ukázala jako nejnáročnější část realizace.
Pro zjednodušení implementace v sobě parser využívá (rozšířené) regulární výrazy kde je to možno. Jejich použití pak
doplňuje například použitím počítadla pro správné ukončování vnitřních prostředí. Dále je používán zásobník, který
umožnuje sledovat aktuální \textit{scope}.

\begin{sloppypar}
Pro parsování meta-bloků (odsazených YAML bloků) byla využita knihovna yaml. Jedná se o aktivně udržovanou, populární
knihovnu (4 až 12 milionů týdenních stažení z npm \cite{npm} během posledního roku) s permisivní open-source licencí,
která také poskytuje TypeScript \textit{types}.
\end{sloppypar}

Parser korektně parsuje každý jednostránkový validní WooWoo dokument. Nechává ale oproti WooWoo specifikaci větší
volnost v prázdných řádcích mezi bloky, objekty a částmi dokumentu. Dále parser podporuje vnořená vnější prostředí, což
opět WooWoo specifikace \cite{woowoo} nedovoluje (narozdíl od objektů). Parser tedy parsuje i některé dokumenty, které
nejsou zcela validní dle WooWoo specifikace. Direktivy \mintinline{text}{.include} aktuálně podporovány nejsou a parser
je považuje za normální text bez speciálního významu.

Parser podporuje jak novou zkrácenou formu vnitřního prostředí se zavináčem, tak starší zkrácenou formu s tečkou. Je
tomu z důvodu použití starší zkrácené formy ve zdrojích existujících studijních textů.

Každý vrchol výsledného AST dědí z abstraktní třídy \mintinline{text}{ASTNode}. Tato abstraktní třída deklaruje druh
vrcholu, jeho rodiče (může být prázdný, což znamená vrchol stromu), seznam jeho dětí (může být prázdný, což znamená list
stromu), jestli je křehký, počáteční a koncovou pozici ve zdroji (řádek, sloupec a odsazení od počátku souboru) a jeho
meta-data. \mintinline{text}{ASTNode} také definuje funkce pro porovnání dvou vrcholů nebo výpis vrcholu do JSON.

Dále mohou vrcholy implementovat interface \mintinline{text}{ValueASTNode} (který deklaruje hodnotu vrcholu) a
rozšiřovat abstraktní třídu \mintinline{text}{VariableASTNode} (která rozšiřuje abstraktní třídu \mintinline{text}
{ASTNode} a deklaruje variantu vrcholu). Variantou vrcholu je myšlen typ částí dokumentu, objektů a vnějších či
vnitřních prostředí (slovo \mintinline{text}{variant} bylo použito namísto slova \mintinline{text}{type}, protože
\mintinline{text}{type} je dle \cite{ts-docs} v jazyce TypeScript klíčové slovo).

Na obrázku \ref{parsovani-ast} je ilustrován (bez detailů jako pozice vrcholu nebo typ/varianta) AST vzniklý parsováním
zdroje \ref{parsovani-ast-zdroj}.

\begin{figure}\centering
    \includegraphics[width=1.0\textwidth]{content/realizace/parsování-ast}
 	\caption[WooWoo AST]{AST vzniklý parsováním WooWoo zdroje~\ref{parsovani-ast-zdroj}}
    \label{parsovani-ast}
\end{figure}

\begin{listing}
    \caption{WooWoo zdroj, jehož AST je ilustrován na obrázku~\ref{parsovani-ast}}
    \label{parsovani-ast-zdroj}
    \begin{minted}[breaklines]{text}
.Chapter Jméno kapitoly
  label: chap-1

text text "text text".emphasize text


.Question:

  text text "text text".emphasize text?

  .solution:

    Řešení
    \end{minted}
\end{listing}


\section{Zobrazení náhledu}
V analýze požadavků \ref{analyza-pozadavku} se nachází řada funkčních požadavků na živý náhled dokumentu. Jedním
požadavkem je \textbf{\ref{F2}}, dalším požadavkem pak \textbf{\ref{F3}}.

Pro zobrazení náhledu bude AST dokumentu získaný parsováním přeložen do HTML, které bude následně zobrazeno v novém
podoknu Atomu. Toto HTML je možno libovolně stylovat díky CSS (případně jeho nadstavby Less, jejíž podporu má v sobě
Atom podle \cite{atom-docs} integrovanou). Je však dobré mít na paměti, že uživatel může chtít použít různá témata
uživatelského prostředí a je tak lepší nechat co nejvíce stylování právě na nich a v případě nutnosti úpravy stylů dělat
takové volby, které budou čitelné napříč populárními tématy prostředí (například tedy není dobré nastavit barvu textu
napevno na bílou či naopak černou). Dále je s tímto HTML možno dál libovolně pracovat za pomoci JavaScriptu a například
tak náhled doplnit o interaktivní prvky.

Živá aktualizace náhledu bude zajištěna díky API Atomu, které dle \cite{atom-docs} umožňuje reagovat na různé změny v
editoru. Při detekci změny pak bude nový obsah opět přeložen na HTML a zobrazen uživateli. Implementace nějakého
chytřejšího systému, kde by se aktualizoval pouze pozměněný obsah, by zřejmě byla složitější a pravděpodobně ani ne
zcela nutná. Je však namístě nějakým způsobem cachovat objekty náročnější na překlad či vykreslení, aby bylo dosaženo
přiměřené rychlosti odezvy.

\subsection{Zobrazování matematických výrazů}
\label{zobrazovani-matematickych-vyrazu}

Dále je kladen funkční požadavek \textbf{\ref{F4}}. Pro zapisování matematických výrazů aktuálně v praxi používaná
WooWoo šablona FIT Template používá \LaTeX{} matematická prostředí. Snažit se přesně zobrazit tyto výrazy není vůbec
jednoduché a je tak namístě použít k jejich zobrazování buď nativní instalaci prostředí \LaTeX{}, nebo jednu z mnohých
knihoven pro zobrazování matematických výrazů na webu.

Výhodou použití nativní instalace je kvalita výstupu a samozřejmě maximální možná podpora \LaTeX{} maker. Nevýhodou
použití nativní instalace je nedostatečná rychlost. Generování obrázku z jediného jednoduchého výrazu trvá řádově
vteřiny (vyzkoušeno na vlastním stroji s procesorem Intel Core i5 7300HQ, 8 GB RAM, operačním systémem Windows 10).
Vzhledem k velmi vysokému množství matematických výrazů ve WooWoo zdrojích existujících studijních textů (řádově vyšší
desítky až stovky výrazů na kapitolu, viz \cite{pkm}) by tak i při spuštění více podprocesů nemohlo být docíleno
akceptovatelné odezvy.

Druhou možností je již zmíněné využití knihoven, jejichž výhodou je oproti nativní instalaci právě rychlost. Další
výhodou těchto knihoven je, že nevznikají externí závislosti (jejichž instalaci si musí koncový uživatel pohlídat).
Nevýhodou knihoven je nižší pokrytí podporovaných \LaTeX{} maker, případně nižší kvalita sazby (která se však mezi
knihovnami výrazně liší také v závislosti na jejich nastavení). Dvěma populárními knihovnami jsou MathJax a KaTeX, které
mezi sebou dále blíže porovnáme. U knihovny MathJax navíc je vhodné oddělit od sebe pro účely porovnání MathJax 2.x a
MathJax 3.x, protože novější velká verze je dle \cite{mathjax3-docs} od základu přepsána a funguje v mnohých ohledech
jinak.

MathJax dle \cite{mathjax} poskytuje kvalitní sazbu matematických výrazů, vstup ve formátech MathML, (La)TeX a ASCIIMath
a výstup ve formátech HTML + CSS, SVG, nebo MathML ve verzi 3.x (a ve verzi 2.x dle \cite{mathjax2-docs} navíc výstupy
PreviewHTML a CommonHTML). MathJax v sobě dále dle \cite{mathjax} má zabudované nástroje pro asistenci nevidomým. Dle
\cite{mathjax3-docs} také podporuje možnost definice vlastních maker. Verze 3.x je dle \cite{mathjax3-docs} rychlejší
než verze 2.x o zhruba 60 až 80 procent, ale neposkytuje zatím tak širokou podporu maker jako starší verze. MathJax je
dle \cite{mathjax3-docs} dostupný (mimo jiné) z npm pod permisivní open-source licencí. HTML výstupy \cite{pkm} z WooWoo
zdroje existujících studijních materiálů využívají pro zobrazování matematických výrazů právě MathJax 2.7.

KaTeX je dle \cite{katex} nejrychlejší knihovna pro zobrazování matematických výrazů na webu, která poskytuje kvalitní
sazbu jejíž rozložení je založeno na rozložení sazby původního \TeX{}. Dle \cite{katex-docs} také podporuje možnost
definování vlastních maker. Za vývojem KaTeX stojí podle \cite{katex} Khan Academy a KaTeX je dostupný (mimo jiné) z npm
pod permisivní open-source licencí. Podle \cite{danmackinlay} KaTeX nenabízí tak širokou podporu matematických maker
jako MathJax.

Vzhledem k nejširší podpoře matematických maker mezi porovnávanými knihovnami bude pro realizaci nejvhodnější použít
MathJax 2.x i na úkor menší rychlosti za předpokladu, že bude postačující pro zobrazení živého náhledu. Navíc tím bude
docíleno parity s existujícími HTML výstupy WooWoo co do podpory zobrazování matematických výrazů. Z dostupných výstupů
MathJax 2.x bude nejlepší použít SVG, protože je dle \cite{mathjax2-docs} druhý nejrychlejší a oproti nejrychlejšímu
výstupu PreviewHTML poskytuje dostatečně kvalitní úroveň sazby.

\subsection{Zobrazování grafických objektů}
\label{zobrazovani-grafickych-objektu}

Posledním funkčním požadavkem z oblasti zobrazení náhledu je \textbf{\ref{F5}}. FIT Template šablona a studijní texty,
které ji využívají, dle \cite{woowoo, pkm} obsahují obrázky popsané \LaTeX{} kódem využívajícím \textit{package} TikZ.

Pro podporu zobrazení těchto objektů se opět nabízí dvě hlavní možnosti – použití nativní instalace \TeX{} distribuce,
nebo použití knihovny. Nativní instalace bude pravděpodobně mít opět rychlostní nevýhodu (a nevýhodu externí
závislosti).

Jediná knihovna pro zobrazování TikZ obrázků na webu, kterou se během rešerše podařilo objevit, je knihovna TikZJax.
Tato knihovna dle \cite{tikzjax-github} využívá originální zdroj \TeX{} zkompilovaný do WebAssembly a její výstup by tak
teoreticky měl být ekvivalentní výstupu nativní instalace. Bohužel je ale aktuálně dle \cite{tikzjax-github} TikZJax
distribuován pouze přes CDN, nikoliv například přes npm a při tvorbě produkčního balíčku v sobě zahrnuje soubory písem z
\LaTeX{} editoru BaKoMa. Tato písma však dle \cite{bakoma-fonts-ctan} neposkytují licenci umožňující volné šíření a
TikZJax tyto soubory nezahrnuje \cite{tikzjax-github} ve svém repozitáři.

Vzhledem k nefunkčnímu požadavku \textbf{\ref{N2}} a vzhledem k tomu, že výstupy některých obrázků obsažených v
existujících studijních textech \cite{pkm} za použití lokálního sestavení TikZJax nedosahují bez potřebných souborů
písem dostačující kvality, bude při realizaci vhodnější použít nativní instalaci \TeX{} distribuce.


\section{Navigace}
\#todo Přehled částí dokumentu a značek, vyhledávání v něm, Symbols View package, ctags, Language Server Protocol, ...


\section{Testování}
Atom poskytuje v sobě zabudovaný systém pro spouštění testů, takzvaný test runner, postavený na testovacím frameworku
Jasmine. Všechny testy musí být ve složce \mintinline{text}{spec} v kořenové složce package a musí mít ve jméně příponu
\mintinline{text}{-spec} (tedy například \mintinline{text}{main-spec.js}). Následně je všechny testy možno spustit z
terminálu příkazem \mintinline{shell}{atom --test spec} (případně lze namísto celé složky s testy použít seznam
konkrétních souborů s testy), nebo přímo z Atomu s pomocí příkazu \mintinline{text}{window:run-package-specs}, za
předpokladu, že je jako aktuální projekt otevřen testovaný package. \cite{atom-man-specs}

Tento Atomem poskytovaný test runner nicméně vnitřně využívá Jasmine ve verzi 1.3. \cite{atom-man-specs} Jedná se o
poněkud starší verzi z roku 2013. \cite{jasmine-old-release} Chceme-li použít novější verzi Jasmine, nebo dokonce úplně
jiný testovací framework, můžeme využít možnosti vytvoření vlastního test runner. Takto vytvořené test runners pak jde
navíc jednoduše poskytnout ostatním pro využití v jejich packages, které si test runner pouze stáhnou jako závislost z
npm a v \mintinline{text}{package.json} řeknou Atomu, že chtějí využít onen konkrétní test runner. \cite{atom-man-specs}

\#todo Výběr konkrétního test runneru

