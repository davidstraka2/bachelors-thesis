Cílem této práce bylo seznámit se s nově vzniklým formátem WooWoo, prozkoumat možnosti moderních editorů či vývojových
prostředí, zejména co se týče jejich rozšiřitelnosti a navrhnout, implementovat a otestovat rozšíření vybraného editoru,
jehož účelem má být právě poskytnutí chybějící okamžité zpětné vazby při tvorbě WooWoo dokumentů a tím zjednodušení
práce s nimi. Zejména jde pak o přehlednou prezentaci logické struktury dokumentů, zobrazování matematických výrazů a
různých grafických objektů (TikZ diagramy), navigaci mezi různými částmi dokumentu a vyhledávání pomocí značek objektů,
a to s využitím stylů založených na existujících šablonách.

Na základě provedené rešerše bylo vytvořeno rozšíření editoru Atom, protože poskytuje ze zkoumaných editorů největší
prostor pro modifikaci a zároveň poskytuje kvalitní dokumentaci a širokou komunitu pro řešení případných problémů.

Toto rozšíření aktuálně nabízí živý náhled otevřeného kusu dokumentu, který přesně a přehledně prezentuje jeho logickou
strukturu. Náhled také zobrazuje v dokumentu obsažené matematické výrazy s využitím knihovny MathJax 2.6, která
poskytuje dostatečnou podporu LaTeX maker využitých v existujících studijních materiálech psaných s pomocí WooWoo.
Náhled dále zobrazuje TikZ diagramy s využitím nativní TeX distribuce. Rozšíření poskytuje okno s přehledem částí a
značek v otevřeném kusu dokumentu a fuzzy vyhledávání v jejich názvech. Bylo využito stylů založených na existujících
šablonách.

Funkčnost rozšíření byla řádně otestována s využitím unit testů, integračních testů a jednoduchého uživatelského
testování. Dále byly prozkoumány možnosti rozšíření editoru pro podporu více šablon a návrh a implementace probíhaly
tak, aby toto případné další rozšíření šlo provést jen s drobnými modifikacemi.

Vzniklé rozšíření usnadní tvorbu studijních textů ve formátu WooWoo díky poskytnutí okamžité zpětné vazby a nabízí navíc
mnoho možností pro další budoucí rozšíření, jako je podpora zobrazení celého dokumentu (ne jen otevřeného kusu),
synchronní posuv náhledu, lišta nástrojů, podpora více šablon a potenciální vznik celého ekosystému šablonových
rozšíření. Zde implementovaný parser by se navíc mohl potencionálně využít i ve zcela jiných projektech, a to i díky
použité permisivní open source licenci a kódu dostupnému na portálu GitHub.
