V analýze požadavků \ref{analyza-pozadavku} se nachází řada funkčních požadavků na živý náhled dokumentu. Jedním
požadavkem je \textbf{\ref{F2}}, dalším požadavkem pak \textbf{\ref{F3}}.

Pro zobrazení náhledu bude AST dokumentu získaný parsováním přeložen do HTML, které bude následně zobrazeno v novém
podoknu Atomu. Toto HTML je možno libovolně stylovat díky CSS (případně jeho nadstavby Less, jejíž podporu má v sobě
Atom podle \cite{atom-docs} integrovanou). Je však dobré mít na paměti, že uživatel může chtít použít různá témata
uživatelského prostředí a je tak lepší nechat co nejvíce stylování právě na nich a v případě nutnosti úpravy stylů dělat
takové volby, které budou čitelné napříč populárními tématy prostředí (například tedy není dobré nastavit barvu textu
napevno na bílou či naopak černou). Dále je s tímto HTML možno dál libovolně pracovat za pomoci JavaScriptu a například
tak náhled doplnit o interaktivní prvky.

Živá aktualizace náhledu bude zajištěna díky API Atomu, které dle \cite{atom-docs} umožňuje reagovat na různé změny v
editoru. Při detekci změny pak bude nový obsah opět přeložen na HTML a zobrazen uživateli. Implementace nějakého
chytřejšího systému, kde by se aktualizoval pouze pozměněný obsah, by zřejmě byla složitější a pravděpodobně ani ne
zcela nutná. Je však namístě nějakým způsobem cachovat objekty náročnější na překlad či vykreslení, aby bylo dosaženo
přiměřené rychlosti odezvy.

\subsection{Zobrazování matematických výrazů}
\label{zobrazovani-matematickych-vyrazu}

Dále je kladen funkční požadavek \textbf{\ref{F4}}. Pro zapisování matematických výrazů aktuálně v praxi používaná
WooWoo šablona FIT Template používá \LaTeX{} matematická prostředí. Snažit se přesně zobrazit tyto výrazy není vůbec
jednoduché a je tak namístě použít k jejich zobrazování buď nativní instalaci prostředí \LaTeX{}, nebo jednu z mnohých
knihoven pro zobrazování matematických výrazů na webu.

Výhodou použití nativní instalace je kvalita výstupu a samozřejmě maximální možná podpora \LaTeX{} maker. Nevýhodou
použití nativní instalace je nedostatečná rychlost. Generování obrázku z jediného jednoduchého výrazu trvá řádově
vteřiny (vyzkoušeno na vlastním stroji s procesorem Intel Core i5 7300HQ, 8 GB RAM, operačním systémem Windows 10).
Vzhledem k velmi vysokému množství matematických výrazů ve WooWoo zdrojích existujících studijních textů (řádově vyšší
desítky až stovky výrazů na kapitolu, viz \cite{pkm}) by tak i při spuštění více podprocesů nemohlo být docíleno
akceptovatelné odezvy.

Druhou možností je již zmíněné využití knihoven, jejichž výhodou je oproti nativní instalaci právě rychlost. Další
výhodou těchto knihoven je, že nevznikají externí závislosti (jejichž instalaci si musí koncový uživatel pohlídat).
Nevýhodou knihoven je nižší pokrytí podporovaných \LaTeX{} maker, případně nižší kvalita sazby (která se však mezi
knihovnami výrazně liší také v závislosti na jejich nastavení). Dvěma populárními knihovnami jsou MathJax a KaTeX, které
mezi sebou dále blíže porovnáme. U knihovny MathJax navíc je vhodné oddělit od sebe pro účely porovnání MathJax 2.x a
MathJax 3.x, protože novější velká verze je dle \cite{mathjax3-docs} od základu přepsána a funguje v mnohých ohledech
jinak.

MathJax dle \cite{mathjax} poskytuje kvalitní sazbu matematických výrazů, vstup ve formátech MathML, (La)TeX a ASCIIMath
a výstup ve formátech HTML + CSS, SVG, nebo MathML ve verzi 3.x (a ve verzi 2.x dle \cite{mathjax2-docs} navíc výstupy
PreviewHTML a CommonHTML). MathJax v sobě dále dle \cite{mathjax} má zabudované nástroje pro asistenci nevidomým. Dle
\cite{mathjax3-docs} také podporuje možnost definice vlastních maker. Verze 3.x je dle \cite{mathjax3-docs} rychlejší
než verze 2.x o zhruba 60 až 80 procent, ale neposkytuje zatím tak širokou podporu maker jako starší verze. MathJax je
dle \cite{mathjax3-docs} dostupný (mimo jiné) z npm pod permisivní open-source licencí. HTML výstupy \cite{pkm} z WooWoo
zdroje existujících studijních materiálů využívají pro zobrazování matematických výrazů právě MathJax 2.7.

KaTeX je dle \cite{katex} nejrychlejší knihovna pro zobrazování matematických výrazů na webu, která poskytuje kvalitní
sazbu jejíž rozložení je založeno na rozložení sazby původního \TeX{}. Dle \cite{katex-docs} také podporuje možnost
definování vlastních maker. Za vývojem KaTeX stojí podle \cite{katex} Khan Academy a KaTeX je dostupný (mimo jiné) z npm
pod permisivní open-source licencí. Podle \cite{danmackinlay} KaTeX nenabízí tak širokou podporu matematických maker
jako MathJax.

Vzhledem k nejširší podpoře matematických maker mezi porovnávanými knihovnami bude pro realizaci nejvhodnější použít
MathJax 2.x i na úkor menší rychlosti za předpokladu, že bude postačující pro zobrazení živého náhledu. Navíc tím bude
docíleno parity s existujícími HTML výstupy WooWoo co do podpory zobrazování matematických výrazů. Z dostupných výstupů
MathJax 2.x bude nejlepší použít SVG, protože je dle \cite{mathjax2-docs} druhý nejrychlejší a oproti nejrychlejšímu
výstupu PreviewHTML poskytuje dostatečně kvalitní úroveň sazby.

\subsection{Zobrazování grafických objektů}
\label{zobrazovani-grafickych-objektu}

Posledním funkčním požadavkem z oblasti zobrazení náhledu je \textbf{\ref{F5}}. FIT Template šablona a studijní texty,
které ji využívají, dle \cite{woowoo, pkm} obsahují obrázky popsané \LaTeX{} kódem využívajícím \textit{package} TikZ.

Pro podporu zobrazení těchto objektů se opět nabízí dvě hlavní možnosti – použití nativní instalace \TeX{} distribuce,
nebo použití knihovny. Nativní instalace bude pravděpodobně mít opět rychlostní nevýhodu (a nevýhodu externí
závislosti).

Jediná knihovna pro zobrazování TikZ obrázků na webu, kterou se během rešerše podařilo objevit, je knihovna TikZJax.
Tato knihovna dle \cite{tikzjax-github} využívá originální zdroj \TeX{} zkompilovaný do WebAssembly a její výstup by tak
teoreticky měl být ekvivalentní výstupu nativní instalace. Bohužel je ale aktuálně dle \cite{tikzjax-github} TikZJax
distribuován pouze přes CDN, nikoliv například přes npm a při tvorbě produkčního balíčku v sobě zahrnuje soubory písem z
\LaTeX{} editoru BaKoMa. Tato písma však dle \cite{bakoma-fonts-ctan} neposkytují licenci umožňující volné šíření a
TikZJax tyto soubory nezahrnuje \cite{tikzjax-github} ve svém repozitáři.

Vzhledem k nefunkčnímu požadavku \textbf{\ref{N2}} a vzhledem k tomu, že výstupy některých obrázků obsažených v
existujících studijních textech \cite{pkm} za použití lokálního sestavení TikZJax nedosahují bez potřebných souborů
písem dostačující kvality, bude při realizaci vhodnější použít nativní instalaci \TeX{} distribuce.
