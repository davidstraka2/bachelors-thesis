V analýze požadavků \ref{analyza-pozadavku} se nachází řada funkčních požadavků na živý náhled dokumentu. Jedním
požadavkem je samotné zobrazení náhledu, dalším požadavkem pak jeho živá aktualizace při úpravě dokumentu.

Pro zobrazení náhledu bude AST dokumentu získaný parsováním přeložen do HTML, které bude následně zobrazeno v novém
podoknu Atomu. Toto HTML je možno libovolně stylovat díky CSS (případně jeho nadstavby Less, jejíž podporu má v sobě
Atom podle \cite{atom-docs} integrovanou). Je však dobré mít na paměti, že uživatel může chtít použít různá témata a je
tak lepší nechat co nejvíce stylování právě na nich a v případě nutnosti úpravy stylů dělat takové volby, které budou
čitelné napříč populárními tématy (například tedy není dobré nastavit barvu textu napevno na bílou či naopak černou).
Dále je s tímto HTML možno dál libovolně pracovat za pomoci JavaScriptu a například tak náhled doplnit o interaktivní
prvky.

Živá aktualizace náhledu bude zajištěna díky API Atomu, které dle \cite{atom-docs} umožňuje reagovat na různé změny v
editoru. Při detekci změny pak bude nový obsah opět přeložen na HTML a zobrazen uživateli. Implementace nějakého
chytřejšího systému, kde by se aktualizoval pouze pozměněný obsah, by zřejmě byla složitější a pravděpodobně ani ne
zcela nutná. Je však namístě nějakým způsobem cachovat objekty náročnější na překlad či vykreslení, aby bylo dosaženo
přiměřené rychlosti odezvy.

\subsection{Zobrazování matematických výrazů}

Dále je kladen požadavek na zobrazování matematických výrazů v náhledu dokumentu. Pro zapisování matematických výrazů
aktuálně v praxi používaná WooWoo šablona FIT Template používá \LaTeX{} matematická prostředí. Snažit se přesně zobrazit
tyto výrazy není vůbec jednoduché a je tak namístě použít k jejich zobrazování buď nativní instalaci prostředí \LaTeX{},
nebo jednu z mnohých knihoven pro zobrazování matematických výrazů na webu.

Výhodou použití nativní instalace je kvalita výstupu a samozřejmě maximální možná podpora \LaTeX{} maker. Nevýhodou
použití nativní instalace je nedostatečná rychlost. Generování obrázku z jediného jednoduchého výrazu trvá řádově
vteřiny (vyzkoušeno na vlastním stroji). Vzhledem k velmi vysokému množství matematických výrazů ve WooWoo zdrojích
existujících studijních textů (řádově vyšší desítky až stovky výrazů na kapitolu, viz \cite{pkm}) by tak i při spuštění
více podprocesů nemohlo být docíleno akceptovatelné odezvy.

Druhou možností je již zmíněné využití knihoven, jejichž výhodou je oproti nativní instalaci právě rychlost. Další
výhodou těchto knihoven je, že nevznikají externí závislosti (jejichž instalaci si musí koncový uživatel pohlídat).
Nevýhodou knihoven je nižší pokrytí podporovaných \LaTeX{} maker, případně nižší kvalita sazby (která se však mezi
knihovnami výrazně liší také v závislosti na jejich nastavení). Dvě populární knihovny jsou MathJax a KaTeX, které
dále blíže porovnáme. U knihovny MathJax navíc je vhodné oddělit od sebe pro účely porovnání MathJax 2.x a MathJax 3.x,
protože novější velká verze je dle \cite{mathjax3-docs} od základu přepsána a funguje v mnohých ohledech jinak.

\#todo KaTeX vs. MathJax, porovnání verzí MathJax

\subsection{Zobrazování grafických objektů}

\#todo TikZ diagramy, TikZJax vs. nativní TeX instalace
