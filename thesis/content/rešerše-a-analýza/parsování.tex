Pro další práci s WooWoo dokumentem (ať už jde o překlad do HTML, vytvoření přehledu určitých prvků a další) je vhodné z
dokumentu vygenerovat abstraktní syntaktický strom (AST) – stromovou strukturu, která popisuje skutečný sémantický
význam zdroje (na rozdíl od derivačního stromu, CST, který zachycuje spíše přesnou syntaktickou strukturu). Pro
vygenerování AST bude nutné vytvořit parser. Toho je možné docílit mnoha způsoby.

Jedním způsobem je využít parser generátorů – programů, kterým je dodán popis gramatiky v nějakém požadovaném formátu,
na jehož základě samy vygenerují výsledný parser. Většina parser generátorů však přijímá nejvýše bezkontextové
gramatiky. Pro případnou podporu kontextově závislých prvků jazyka je možno vygenerovaný parser vlastními silami
upravit, což však vyžaduje pokročilé znalosti v oblasti tvorby parserů.

Pro generování AST proto vytvoříme vlastní parser, který bude pro zjednodušení implementace vnitřně využívat
(rozšířených) regulárních výrazů pro parsování těch částí WooWoo formátu, kde to bude možné. Pro parsování neregulárních
bezkontextových prvků WooWoo formátu (například správné určení začínajících a ukončovacích tokenů vnitřních prostředí)
nebo kontextově závislých prvků (například změna významu na základě úrovně odsazení) bude potřeba použití (rozšířených)
regulárních výrazů doplnit o vlastní funkcionalitu, jako různá počítadla, zásobníky apod.

AST obecně nemají nějaký standardně používaný formát (viz \cite{ast-explorer}), liší se parser od parseru, ale v mnoha
existujících AST (pro jiné jazyky) si můžeme všimnout častých prvků, které bude vhodné v našem AST také zahrnout. Jedná
se například o pozici (řádek, sloupec a odsazení od začátku souboru) začátku a konce vrcholu stromu. Tato pozice umožní
například přesnou navigaci mezi částmi dokumentu, synchronní posuv náhledu se zdrojem a mnoho dalších funkcionalit.
