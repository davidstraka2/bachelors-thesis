Atom poskytuje v sobě zabudovaný systém pro spouštění testů, takzvaný test runner, postavený na testovacím frameworku
Jasmine. Všechny testy musí být ve složce \mintinline{text}{spec} v kořenové složce package a musí mít ve jméně příponu
\mintinline{text}{-spec} (tedy například \mintinline{text}{main-spec.js}). Následně je všechny testy možno spustit z
terminálu příkazem \mintinline{shell}{atom --test spec} (případně lze namísto celé složky s testy použít seznam
konkrétních souborů s testy), nebo přímo z Atomu s pomocí příkazu \mintinline{text}{window:run-package-specs}, za
předpokladu, že je jako aktuální projekt otevřen testovaný package. \cite{atom-man-specs}

Tento Atomem poskytovaný test runner nicméně vnitřně využívá Jasmine ve verzi 1.3. \cite{atom-man-specs} Jedná se o
poněkud starší verzi z roku 2013. \cite{jasmine-old-release} Chceme-li použít novější verzi Jasmine, nebo dokonce úplně
jiný testovací framework, můžeme využít možnosti vytvoření vlastního test runner. Takto vytvořené test runners pak jde
navíc jednoduše poskytnout ostatním pro využití v jejich packages, které si test runner pouze stáhnou jako závislost z
npm a v \mintinline{text}{package.json} řeknou Atomu, že chtějí využít onen konkrétní test runner. \cite{atom-man-specs}

\#todo Výběr konkrétního test runneru
