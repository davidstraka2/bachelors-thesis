Dle \cite{atom-docs} v sobě Atom poskytuje integrovaný systém pro spouštění testů, takzvaný \textit{test runner},
postavený na testovacím frameworku Jasmine. Dále \cite{atom-docs} uvádí, že všechny testy musí být ve složce
\mintinline{text}{spec} v kořenové složce \textit{package} a musí mít ve jméně příponu \mintinline{text}{-spec} (tedy
například \mintinline{text}{main-spec.js}). Následně je podle \cite{atom-docs} možné všechny testy spustit z terminálu
příkazem \mintinline[breaklines]{shell}{atom --test spec} (případně lze namísto celé složky s testy použít seznam
konkrétních souborů s testy), nebo přímo z Atomu s pomocí příkazu \mintinline[breaklines]{text}
{window:run-package-specs}, za předpokladu, že je jako aktuální projekt otevřen právě testovaný \textit{package}.

Tento Atomem poskytovaný \textit{test runner} nicméně dle \cite{atom-docs} vnitřně využívá Jasmine ve verzi 1.3. V
\cite{jasmine-releases} najdeme, že se jedná o poněkud starší verzi z roku 2013. Chceme-li použít novější verzi Jasmine,
nebo dokonce úplně jiný testovací framework, můžeme podle \cite{atom-docs} využít možnosti vytvoření vlastního
\textit{test runner}. Takto vytvořené \textit{test runners} pak navíc dle \cite{atom-docs} jde jednoduše poskytnout
ostatním vývojářům pro využití v jejich \textit{packages}, které si konkrétní \textit{test runner} pouze stáhnou jako
závislost z npm a v \mintinline{text}{package.json} řeknou Atomu, že chtějí využít onen konkrétní \textit{test runner}.

\subsection{Výběr komunitního \textit{test runner}}
\label{vyber-komunitniho-test-runner}

Protože vytvoření vlastního \textit{test runner} nemusí být zrovna jednoduché, není to cílem této práce a zároveň
už existuje více \textit{test runners}, mezi kterými si můžeme vybrat, porovnáme několik těchto komunitních \textit{
test runners} a pro realizaci vybereme jeden z nich.

Použití testovacího frameworku Jest umožňuje \textit{test runner} atom-test-runner-jest \cite{test-runner-jest-github}.
Tento \textit{test runner} se ale nezdá být aktivně udržován (poslední commit je dle \cite{test-runner-jest-github} z
června 2018), \textit{issue tracker} obsahuje dle \cite{test-runner-jest-github} dva otevřené issues z celkového počtu
tří issues, všechny také z roku 2018. Komunita kolem tohoto \textit{test runner} se zdá malá (jednotky až nízké desítky
týdenních stažení během posledního roku z npm \cite{npm}).

Použití testovacího frameworku Mocha umožňuje \textit{test runner} atom-mocha-test-runner
\cite{test-runner-mocha-github}. Tento \textit{test runner} je na tom z hlediska aktivity údržby lépe než
atom-test-runner-jest. Poslední commit je dle \cite{test-runner-mocha-github} z prosince 2020 (je však pouze o
poloautomatický update závislostí; poslední commit bez těchto prostých aktualizací je z června 2019), \textit
{issue tracker} dle \cite{test-runner-mocha-github} neobsahuje žádný otevřený issue z celkem dvou issues (z nichž
poslední je z prosince 2020). Komunita kolem tohoto \textit{test runner} se zdá být větší než u atom-test-runner-jest
(vyšší stovky až přes tisíc týdenních stažení během posledního roku z npm \cite{npm}). Tento \textit{test runner} navíc
podle \cite{test-runner-mocha-github} spravují přímo vývojáři Atomu.

Použití aktuální verze frameworku Jasmine (což je v době psaní verze 3.7) umožňuje \textit{test runner}
atom-jasmine3-test-runner \cite{test-runner-jasmine3-github}. Tento \textit{test runner} se na tom zdá být nejlépe jak z
hlediska aktivní údržby, tak z hlediska velikosti komunity kolem. Nové commity dle \cite{test-runner-jasmine3-github}
přibývají každý týden, \textit{issue tracker} dle \cite{test-runner-jasmine3-github} neobsahuje žádný otevřený issue z
celkových 12 issues (z nichž poslední je z ledna tohoto roku), počty týdenních stažení z npm se dle \cite{npm} pohybují
kolem vyšších stovek až nižších jednotek tisíc během posledního roku. Autor tohoto \textit{test runner} je navíc velmi
aktivní i v dalších komunitních projektech kolem Atomu (například také spravuje \cite{github-action-setup-atom} GitHub
\textit{action} pro použití Atomu v nástroji pro kontinuální integraci GitHub Actions) a Jasmine. Tento
\textit{test runner} je podle \cite{test-runner-jasmine3-github} také využíván pro testování například v
\textit{package} autocomplete-plus \cite{atom-package-autocomplete-plus}, což je \textit{core package} Atomu, nebo také
v \textit{package} Hydrogen \cite{atom-package-hydrogen}, oblíbeném prostředí pro práci s Jupyter notebooky.

Podobných \textit{test runners} existuje mnohem více, ale převažuje mezi nimi problém neaktivní údržby, malé komunity,
případně i otevřených issues s problematickými chybami. Z výše popsaných \textit{test runners} se jeví
atom-jasmine3-test-runner jako nejlepší volba, díky aktivní údržbě a větší komunitě, která tento \textit{test runner}
využívá.
