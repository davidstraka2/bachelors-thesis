Atom \textit{packages} je dle \cite{atom-docs} možno tvořit v jazycích JavaScript nebo CoffeeScript. Další možností je
samozřejmě napsat \textit{package} v jiném jazyce, který je možno do jednoho z Atomem podporovaných jazyků přeložit.

Dle \cite{atom-docs} je potřeba, aby \textit{package} obsahoval v kořenovém adresáři meta-soubor \mintinline{text}
{package.json}. V tomto souboru bude dle \cite{atom-docs} uveden například unikátní název \textit{package} a jeho verze,
dále zde může být uveden jeho popis, autor a kontakt na něj, odkaz na \textit{issue tracker}, odkaz na repozitář se
zdrojem či jiné webové stránky a mnoho dalších informací ve stylu npm \textit{packages}. Dále zde dle \cite{atom-docs}
mohou být uvedeny různé závislosti z npm, které \textit{package} potřebuje ke svému běhu (nebo pouze k vývoji). Tyto
závislosti jsou pak dle \cite{atom-docs} automaticky nainstalovány při instalaci samotného \textit{package}. Také zde
podle \cite{atom-docs} může být uvedena cesta k \textit{main} souboru \textit{package} (pokud není uvedena, použije se
výchozí možnost \mintinline{text}{index.coffee} nebo \mintinline{text}{index.js}) nebo seznam příkazů, které \textit
{package} v Atomu aktivují.

Hlavní modul (\textit{main} soubor) \textit{package} může dle \cite{atom-docs} implementovat následující funkce, jejichž
volání spravuje sám Atom:
\begin{itemize}
    \begin{sloppypar}
    \item \mintinline{js}{activate(state)} – tato funkce je volána, když je \textit{package} aktivován (například po
        zavolání příkazu, který je v \mintinline{text}{package.json} definován jako aktivační) a jejím prvním argumentem
        je stav z posledního volání \mintinline{js}{serialize};
    \end{sloppypar}
    \item \mintinline{js}{initialize(state)} – tato funkce je podobná funkci \mintinline{js}{activate(state)}, je ale
        volána dříve, ještě před deserializací (při volání \mintinline{js}{activate(state)} je zajištěno, že je
        pracovní prostředí Atomu připraveno);
    \begin{sloppypar}
    \item \mintinline{js}{serialize} – tato fukce je volána, když je zavíráno okno Atomu a vrací JSON popisující stav
        \mintinline{text}{package}, který je později podán funkci \mintinline{js}{activate(state)} při opětovné
        aktivaci;
    \end{sloppypar}
    \item \mintinline{js}{deactivate} – tato funkce je volána, když je zavírano okno Atomu nebo když je
        \mintinline{text}{package} deaktivován.
\end{itemize}
Dále dle \cite{atom-docs} může hlavní modul exportovat \mintinline{js}{config} objekt definující položky nastavení
\textit{package}, které si může uživatel přizpůsobit (jejich aktuální hodnotu pak lze získat opět pomocí API). Kód
\textit{package} má dle \cite{atom-docs} vždy k dispozici globální proměnnou \mintinline{js}{atom}, která poskytuje
přístup k API Atomu. Soubory s JavaScript/CoffeeScript kódem patří dle \cite{atom-docs} do složky \mintinline{text}
{lib}, soubory s testy pak do složky \mintinline{text}{spec}.

Dále může \textit{package} dle \cite{atom-docs} obsahovat různé složky s meta-soubory, které definují určitou
funkcionalitu \textit{package} v rámci Atomu (jedná se o jednodušší alternativu k použití API, které má ale samozřejmě
mnohem širší možnosti). Takovými složkami jsou dle \cite{atom-docs} například složky \mintinline{text}{grammars} s
\textit{grammar} soubory přidávajícími do Atomu podporu zvýrazňování syntaxe pro nové jazyky, \mintinline{text}{keymaps}
se soubory definujícími klávesové zkratky příkazů \textit{package} (ty si pak může koncový uživatel upravit v
nastaveních), \mintinline{text}{menus} se soubory přidávajícími položky do hlavního menu a kontextového menu Atomu,
\mintinline{text}{snippets} se soubory definujícími úryvky kódu pro rychlé použití nebo \mintinline{text}{styles} s CSS
či Less soubory upravujícími styly Atomu.

Atom také dle \cite{atom-docs} umožňuje poskytnout či použít \textit{services}, verzovaná API, umožňující komunikaci
napříč více \textit{packages}. Tohoto by v budoucnu mohlo být využito pro rozdělení \textit{package} do více částí,
kde hlavní \textit{package} by například poskytnul \textit{services} pro přidání nových funkcí či tříd pro překlad prvků
WooWoo AST do HTML. Takto by mohl vzniknout celý ekosystém \textit{packages} pro podporu WooWoo, kde by například jeden
\textit{package} přidal podporu pro zobrazení matematických výrazů, další \textit{package} by přidal podporu pro
zobrazení TikZ obrázků a jiný \textit{package}, který by definoval šablonu, by řekl, že závisí na hlavním \textit
{package} a dalších konkrétních \textit{packages} přidávajících podporu zobrazení dalších prvků. Tento \textit{package}
by také řekl, jak konkrétně šablona, kterou definuje, nazývá a zobrazuje různé typy prostředí, objektů a částí
dokumentu. Takto by v budoucnu mohlo být docíleno modulární podpory více WooWoo šablon.

Vydaný \textit{package} si podle \cite{atom-docs} uživatelé mohou nainstalovat ze zabudovaného správce \textit{packages}
(nebo z terminálu s pomocí nástroje apm, který je také součástí Atomu). Aby bylo možno \textit{package} publikovat, musí
dle \cite{atom-docs} jeho výstupní JavaScript či CoffeeScript soubory (spolu s dalšími podpůrnými soubory) být
zveřejněny v GitHub repozitáři. Následně je dle \cite{atom-docs} za pomoci nástroje apm možno publikovat buď aktuální
obsah hlavní větve, kdy apm s pomocí GitHub API sám vytvoří nový release/tag (verze je brána z meta-souboru \mintinline
{text}{package.json}), nebo je možno vytvořit release/tag vlastními silami a následně nástroji apm říct,
který konkrétní tag má publikovat.
