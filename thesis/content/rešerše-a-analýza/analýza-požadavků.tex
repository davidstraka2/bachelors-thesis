Na editor je kladena řada funkčních i nefunkčních požadavků, které jsou blíže popsány níže.

Mezi funkční požadavky patří následující:

\begin{enumerate}[label=\textbf{F\arabic*}, ref=F\arabic*]
    \item zvýrazňování syntaxe WooWoo souborů,
    \item zobrazení náhledu dokumentu nebo jeho části,
    \item živá aktualizace náhledu dle aktuálně se měnícího obsahu dokumentu,
    \item zobrazení matematických výrazů v náhledu dokumentu,
    \item zobrazení přehledu částí dokumentu,
    \item zobrazení matematických výrazů v názvech částí v přehledu částí dokumentu,
    \item zobrazení přehledu značek dokumentu,
    \item vyhledávání v přehledu částí dokumentu,
    \item vyhledávání v přehledu značek dokumentu.
\end{enumerate}

Mezi nefunkční požadavky pak patří zejména:

\begin{enumerate}[label=\textbf{N\arabic*}, ref=N\arabic*]
    \item multiplatformnost (editor by měl fungovat na aktuálních verzích nejpoužívanějších desktopových operačních
        systémů – Windows 10, macOS 11 a Linuxové systémy),
    \item nezávislost na internetovém připojení,
    \item možnost jednoduchého rozšíření pro podporu dalších šablon.
\end{enumerate}
