Na editor je kladena řada funkčních i nefunkčních požadavků, které jsou blíže popsány níže.

Mezi funkční požadavky patří následující:

\begin{enumerate}[label=\textbf{F\arabic*}, ref=F\arabic*]
    \item \label{F1} zvýrazňování syntaxe WooWoo formátu,
    \item \label{F2} zobrazení náhledu dokumentu nebo jeho části,
    \item \label{F3} živá aktualizace náhledu dle aktuálně se měnícího obsahu dokumentu,
    \item \label{F4} zobrazení matematických výrazů v náhledu dokumentu,
    \item \label{F5} zobrazení grafických objektů v náhledu dokumentu,
    \item \label{F6} zobrazení obsahu dokumentu,
    \item \label{F7} zobrazení vnitřních prostředí v názvech částí v obsahu dokumentu,
    \item \label{F8} zobrazení přehledu značek dokumentu,
    \item \label{F9} vyhledávání v přehledu částí dokumentu,
    \item \label{F10} vyhledávání v přehledu značek dokumentu.
\end{enumerate}

Mezi nefunkční požadavky pak patří zejména:

\begin{enumerate}[label=\textbf{N\arabic*}, ref=N\arabic*]
    \item \label{N1} multiplatformnost (editor by měl fungovat na aktuálních verzích nejpoužívanějších desktopových
        operačních systémů – Windows 10, macOS 10.15 a Linuxové systémy),
    \item \label{N2} nezávislost na internetovém připojení,
    \item \label{N3} možnost jednoduchého rozšíření pro podporu dalších šablon.
\end{enumerate}
