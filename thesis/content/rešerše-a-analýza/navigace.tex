V analýze požadavků \ref{analyza-pozadavku} se nachází řada funkčních požadavků na navigaci v dokumentu. Tyto funkční
požadavky lze rozdělit na dvě skupiny – požadavky na navigaci v částech dokumentu a požadavky na navigaci ve značkách v
dokumentu.

Navigaci v částech dokumentu by bylo možné realizovat jako navigační okno ve stylu těch, která nabízí například
prohlížeče PDF dokumentů nebo textové procesory (např. Microsoft Word \cite{word-docs}). Uživateli by se tak zobrazil
víceúrovňový seznam, kde každá úroveň obsahuje všechny části dokumentu dané úrovně seřazeny za sebou tak, jak se v
dokumentu vyskytují. Nad seznamem by bylo umístěno vyhledávací pole, s jehož pomocí by uživatel mohl vyfiltrovat pouze
relevantní části dokumentu. Vytvořit takový seznam půjde díky AST vzniklému parsováním. Pro zobrazení každé položky
seznamu, tedy názvu části dokumentu, by mohla být zavolána stejná funkce, která jej zobrazí v náhledu dokumentu. Tím
bude postaráno i o zobrazení případných vnitřních prostředí v názvu dokumentu. Při kliknutí na položku seznamu by pak
došlo ke skoku na odpovídající řádek v editoru (řádek, kde začíná část dokumentu, také umožní určit AST).

Pro zobrazení seznamu značek obsažených v dokumentu s možností fuzzy vyhledávání, skokem na vybranou značku ze seznamu
do dokumentu nebo z reference na značku se nabízí využít \textit{core package} Atomu Symbols View. Tento
\textit{package} dle \cite{atom-package-symbols-view} umožňuje takovou navigaci mezi symboly dokumentu, které je nutné
popsat v \textit{tags} souboru ve formátu kompatibilním s nástrojem Ctags. Ctags z pochopitelných důvodů neumí tento
\textit{tags} soubor vygenerovat pro WooWoo dokumenty. Toto by tedy bylo nutné naimplementovat v našem rozšíření, nebo
případně využít jednoho ze způsobů jak rozšířit Ctags o vlastní parser (viz \cite{ctags-docs}).

Implementace generování \textit{tags} souboru se zdá jako relativně jednoduché řešení, protože Ctags formáty mají
podle \cite{ctags-docs} jednoduchou syntaxi. Navíc všechny informace, které je v \textit{tags} souboru nutno uvést,
získáme z AST vytvořeného při parsování, které budeme implementovat už jen kvůli překladu WooWoo do HTML pro zobrazení
náhledu. Bohužel Symbols View dle \cite{atom-package-symbols-view-github} při zobrazení symbolů v aktuálně otevřeném
souboru pro jejich získání sám volá nástroj Ctags, namísto toho, aby se je pokusil získat z již existujícího
\textit{tags} souboru (toto se liší od chování při zobrazení symbolů v celém projektu, kdy Symbols View nejprve zkouší
číst existující \textit{tags} soubor) a chybí tak způsob, jak mu náš vlastní soubor předat. Vzhledem k velkému množství
značek ve WooWoo zdrojích existujících studijních textů je zrovna možnost zobrazení symbolů pouze z aktuálního souboru
důležitá.

Existuje také \textit{package} atom-ctags, fork Symbols View \textit{package}, který Symbols View v mnoha směrech
rozšířil a pozměnil \cite{atom-package-atom-ctags-github}. Nabízí se tedy otázka, zdali by v tomto směru nemohl fungovat
lépe. Tento fork ale obsahuje dle jeho \textit{issue tracker} \cite{atom-package-atom-ctags-github} nemalé množství
problematických chyb a jeho vývoj či údržba je dle \cite{atom-package-atom-ctags-github} neaktivní od roku 2017.

\begin{sloppypar}
Dalšími možnostmi je tedy vytvořit nový fork Symbols View \textit{package} a upravit jej dle vlastních potřeb, nebo
rozšířit Ctags o vlastní parser (což by ale možná také nakonec vyžadovalo drobné úpravy Symbols View \textit{package}).
Zde už se však zdá jednodušší vytvořit si vlastní přehled značek v otevřeném souboru, kde implementace bude převážně
stejná či hodně podobná jako implementace zobrazení přehledu částí dokumentu.
\end{sloppypar}

Alternativně by ještě bylo možné vytvořit \textit{language server} pro komunikaci s Atomem přes Language Server
Protocol (viz \cite{lsp-docs}), kterou na straně Atomu umožňuje knihovna Atom Language Server Protocol Client
\cite{atom-lsp-client-github}. Tento přístup by dle \cite{atom-lsp-client-github} umožnil mimo jiné právě navigaci z
reference značky v dokumentu na její definici. Navíc by takto implementovanou funkcionalitu podle \cite{lsp-docs} bylo
možno využít i v jiných editorech, které podporují Language Server Protocol (v současnosti dle \cite{lsp-docs} například
VS Code, Visual Studio nebo Brackets, ve formě rozšíření také například Eclipse, Sublime Text, emacs nebo vim). Tento
přístup by ale dle \cite{atom-lsp-client-github} sám o sobě neposkytl seznam všech značek a možnost vyhledávání v něm.
Dále by pravděpodobně vyžadoval komplexnější implementaci.
