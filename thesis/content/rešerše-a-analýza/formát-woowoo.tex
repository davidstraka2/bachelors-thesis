WooWoo je nový formát pro tvorbu (zejména) studijních textů, vzniklý z iniciativy Ing. Tomáše Kalvody, Ph.D. z Katedry
aplikované matematiky na Fakultě informačních technologií Českého vysokého učení technického v Praze. V současnosti je
tento formát využívaný pro tvorbu studijních textů pro dva předměty, s nimiž se studenti setkají hned v prvním semestru
bakalářského studia (aktuálně dobíhající akreditace) – Základy matematické analýzy (BI-ZMA) a Přípravný kurz matematiky
(BI-PKM). Tento formát se snaží vyřešit problém oddělení samotného obsahu dokumentů od informací o způsobu jejich
zobrazení a spolu s tím problém kvalitního víceformátového (HTML a PDF; v budoucnu plánován také EPUB) výstupu z jednoho
zdroje.

Každý WooWoo dokument může být rozdělen do více částí. Každou definici části dokumentu tvoří na jednom řádku její typ,
nadpis a následně může na dalších přímo navazujících řádcích volitelně obsahovat meta-blok (odsazený YAML blok). Typ
části dokumentu je uvozen tečkou a začíná velkým písmenem; od nadpisu je oddělen mezerou.

Pro ilustraci viz zobecněný formát definice části dokumentu \ref{format-woowoo-cast} nebo definice kapitoly studijního
textu BI-PKM i s meta-blokem, která je následována blokem textu (tento studijní text využívá existující FIT Template
šablony) \ref{format-woowoo-pkm-kapitola}.

\begin{listing}
    \caption{Obecný formát definice části WooWoo dokumentu}
    \label{format-woowoo-cast}
    \begin{minted}[breaklines]{text}
.MojeCast Nadpis
  ...
  volitelný YAML meta-blok
  ...
    \end{minted}
\end{listing}

\begin{listing}
    \caption{Část dokumentu ve zdroji studijního textu k BI-PKM \cite{pkm}}
    \label{format-woowoo-pkm-kapitola}
    \begin{minted}[breaklines]{text}
.Chapter Úvod
  label: chap-intro

Než se pustíme do hlavního výkladu, je vhodné čtenáře nejprve
seznámit s cíli a smyslem tohoto kurzu, resp. studijního textu.
    \end{minted}
\end{listing}

\#todo Objekty, vnitřní a vnější prostředí, bloky, šablony, FIT Template
