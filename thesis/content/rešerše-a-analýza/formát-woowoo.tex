WooWoo je nový formát pro tvorbu (zejména) studijních textů, vzniklý z iniciativy Ing. Tomáše Kalvody, Ph.D. z Katedry
aplikované matematiky na Fakultě informačních technologií Českého vysokého učení technického v Praze. V současnosti je
tento formát využívaný pro tvorbu studijních textů pro dva předměty, s nimiž se studenti setkávají hned v prvním
semestru bakalářského studia (aktuálně dobíhající akreditace) – Základy matematické analýzy (BI-ZMA) a Přípravný kurz
matematiky (BI-PKM).

Primárním cílem tohoto formátu je dle \cite{woowoo} kompletní oddělení zdroje dokumentů s možnými meta-informacemi od
informací o způsobu jejich zobrazení. Zdroj dokumentu a případné meta-informace chce podle \cite{woowoo} formát
prezentovat v člověkem čitelné podobě. Dále se tento formát v kombinaci s šablonami snaží vyřešit problém kvalitního
víceformátového (HTML a PDF; v budoucnu plánován také EPUB) výstupu z jednoho zdroje.

WooWoo dokument se dle \cite{woowoo} může dělit do více částí a může obsahovat bloky a objekty (všechny tyto konstrukty
jsou podrobněji popsány v nadcházejících podsekcích). Části dokumentu, bloky i objekty mohou být dle \cite{woowoo} dále
doplněny o meta-bloky (odsazenými YAML bloky).

Konkrétní dostupné typy částí dokumentu, objektů a dalších konstruktů jsou dle \cite{woowoo} závislé na použité šabloně.
Stejně tak dle \cite{woowoo} šablona určuje způsob nakládání s informacemi z meta-bloků.

\subsection{Části dokumentu}

Každý WooWoo dokument může dle \cite{woowoo} být rozdělen do více částí. Podle \cite{woowoo} každou definici části
dokumentu tvoří na jednom řádku její typ, nadpis a následně může na dalších přímo navazujících řádcích volitelně
obsahovat meta-blok (odsazený YAML blok). Typ části dokumentu je dle \cite{woowoo} uvozen tečkou a začíná velkým
písmenem; od nadpisu je oddělen mezerou. Definice části vždy začíná na začátku řádku, nemůže mít před sebou ani bílé
znaky (mezery).

Pro ilustraci viz zobecněný formát definice části dokumentu \ref{format-woowoo-cast} nebo definice kapitoly studijního
textu BI-PKM i s meta-blokem, která je následována blokem textu (tento studijní text využívá existující FIT Template
šablony) \ref{format-woowoo-pkm-kapitola}.

\begin{listing}
    \caption{Obecný formát definice části WooWoo dokumentu}
    \label{format-woowoo-cast}
    \begin{minted}[breaklines]{text}
.MojeCast Nadpis
  ...
  volitelný YAML meta-blok
  ...
    \end{minted}
\end{listing}

\begin{listing}
    \caption{Část dokumentu ve zdroji studijního textu k BI-PKM \cite{pkm}}
    \label{format-woowoo-pkm-kapitola}
    \begin{minted}[breaklines]{text}
.Chapter Úvod
  label: chap-intro

Než se pustíme do hlavního výkladu, je vhodné čtenáře nejprve
seznámit s cíli a smyslem tohoto kurzu, resp. studijního textu.
    \end{minted}
\end{listing}

\subsection{Objekty}

Každý WooWoo dokument může dle \cite{woowoo} obsahovat libovolné množství objektů. Každou definici objektu tvoří dle
\cite{woowoo} na jednom řádku jeho typ a následně může na dalších přímo navazujících řádcích volitelně obsahovat
meta-blok (odsazený YAML blok). Typ objektu je dle \cite{woowoo} uvozen tečkou, začíná velkým písmenem a je následován
dvojtečkou. Tělo objektu může obsahovat odsazené bloky, oddělené mezi sebou alespoň dvěma prázdnými řádky.

Pro ilustraci viz zobecněný formát definice objektu \ref{format-woowoo-objekt} nebo definice konkrétního typu objektu ze
studijního textu BI-PKM i s jeho meta-blokem, která je následována bloky \ref{format-woowoo-pkm-otazka}.

\begin{listing}
    \caption{Obecný formát definice objektu WooWoo dokumentu}
    \label{format-woowoo-objekt}
    \begin{minted}[breaklines]{text}
.MujObjekt:
  ...
  volitelný YAML meta-blok
  ...

  odsazený blok


  odsazený blok


  ...
    \end{minted}
\end{listing}

\begin{listing}
    \caption{Objekt ve zdroji studijního textu k BI-PKM \cite{pkm}}
    \label{format-woowoo-pkm-otazka}
    \begin{minted}[breaklines]{text}
.Question:
  difficulty: 2

  Nechť $a>b>0$. O číslech $a$ a $b$ říkáme, že jsou ve zlatém
  poměru "Hodnota tohoto poměru se také někdy nazývá zlatým řezem.".footnote, pokud poměr $a+b$ ku $a$ je stejný
  jako $a$ ku $b$. Jaký je tento poměr, tedy
  $\varphi = \frac{a}{b}$?

  .solution:

    $\displaystyle\varphi = \frac{1+\sqrt{5}}{2}$.
    \end{minted}
\end{listing}

\subsection{Bloky}

Blok je dle \cite{woowoo} série po sobě jdoucích, jednotně odsazených řádků textu (textový blok) a vnějších prostředí,
oddělených jedním prázdným řádkem. Textový blok i vnější prostředí mohou být podle \cite{woowoo} na dalších přímo
navazujících řádcích doplněny meta-blokem (odsazeným YAML blokem). Dle \cite{woowoo} v sobě textový blok dále může
obsahovat vnitřní prostředí. Vnější a vnitřní prostředí jsou podrobněji popsány v nadcházejících podsekcích.

Pro ilustraci viz blok ze studijního textu BI-PKM \ref{format-woowoo-pkm-blok}. Uvedený blok obsahuje textový blok (s
vnitřním prostředím), následovaný zkráceným vnějším prostředím bez hlavičky (které se ve FIT Template šabloně chová jako
vnější prostředí reprezentující blokovou matematiku), následované dalším textovým blokem a vnějším prostředím
\mintinline{text}{.solution}.

\begin{listing}
    \caption{Blok ve zdroji studijního textu k BI-PKM \cite{pkm}}
    \label{format-woowoo-pkm-blok}
    \begin{minted}[breaklines]{text}
Čemu se rovná součet prvních $n$ lichých přirozených čísel? Tj.
čemu se rovná součet

  1 + 3 + 5 + \cdots + (2n-1) = \sum_{j=1}^n (2j-1), \quad n \in
  \mathbb{N}.

Své tvrzení dokažte.

.solution:

  $\sum_{j=1}^n (2j - 1) = n^2$. Tento vztah laskavý čtenář
  snadno dokáže pomocí matematické indukce.
    \end{minted}
\end{listing}

\subsection{Vnější prostředí}

Každý blok může dle \cite{woowoo} obsahovat libovolné množství vnějších prostředí. Syntaxe vnějšího prostředí je dle
\cite{woowoo} podobná syntaxi objektů, kde jediným rozdílem je, že typ vnějšího prostředí začíná malým písmenem a tělo
obsahuje pouze jeden blok textu.

Hlavičku vnějšího prostředí je dle \cite{woowoo} možno vynechat. Jedná se pak v podstatě o speciální blok, který je
odsazen více než blok předchozí. Chování tohoho zkráceného vnějšího prostředí či speciálně odsazeného bloku pak dle
\cite{woowoo} určuje konkrétní použitá šablona jako zkratku nějakého z vnějších prostředí.

Definice vnějšího prostředí může být dle \cite{woowoo} uvozena namísto tečky vykřičníkem. V takovém případě se podle
\cite{woowoo} jedná o křehké prostředí a neprovádí se žádné další parsování nebo transformace jeho obsahu.

Pro ilustraci viz zobecněný formát definice vnějšího prostředí \ref{format-woowoo-outer-env}, definice konkrétního typu
vnějšího prostředí ze studijního textu BI-PKM i s jeho meta-blokem, která je následována blokem textu (tento
studijní text využívá existující FIT Template šablony) \ref{format-woowoo-pkm-quote}, případně definice konkrétního typu
křehkého vnějšího prostředí ze stejného zdroje \ref{format-woowoo-pkm-tikz}.

\begin{listing}
    \caption{Obecný formát definice vnějšího prostředí WooWoo dokumentu}
    \label{format-woowoo-outer-env}
    \begin{minted}[breaklines]{text}
.mojeVnejsiProstredi:
  ...
  volitelný YAML meta-blok
  ...

  odsazený blok
    \end{minted}
\end{listing}

\begin{listing}
    \caption{Vnější prostředí ve zdroji studijního textu k BI-PKM \cite{pkm}}
    \label{format-woowoo-pkm-quote}
    \begin{minted}[breaklines]{text}
.quote:
  author: 'Nápis na vstupu do Platónovy Akademie'

  Bez znalosti geometrie sem nikdo nevstupuj.
    \end{minted}
\end{listing}

\begin{listing}
    \caption{Křehké vnější prostředí ve zdroji studijního textu k BI-PKM~\cite{pkm}}
    \label{format-woowoo-pkm-tikz}
    \begin{minted}[breaklines]{text}
!tikz:
  options: 'scale=1.3'
  filename: fig-komplexni-rovina

  \def\a{1.2}
  \def\b{.8}
  \draw[thick,->] (-2,0) -- (2,0) node[above] {$\Re$};
  \draw[thick,->] (0,-2) -- (0,2) node[left] {$\Im$};
  \draw[gray,dashed] (\a,-2) -- (\a,2) (-2,\b) -- (2,\b);
  \fill[black] (\a,\b) circle (0.05) node[anchor=south west]
    {$a+\ii b$};
  \draw[thick] (.05,\b) -- (-.05,\b) node[anchor=south east]
    {$b$};
  \draw[thick] (\a,.05) -- (\a,-.05) node[anchor=north west]
    {$a$};
    \end{minted}
\end{listing}

\subsection{Vnitřní prostředí}

Každý textový blok v sobě dle \cite{woowoo} může obsahovat libovolné množství vnitřních prostředí. Vnitřní prostředí
musí být definováno na jednom řádku. Vnitřní prostědí lze dle \cite{woowoo} definovat dvěma způsoby: krátkou formou a
verbózní formou. Typ vnitřního prostředí začíná dle \cite{woowoo} v obou případech malým písmenem.

Krátká forma vnitřního prostředí má dle \cite{woowoo} prefixovou podobu. Definici takového prostředí tvoří dle
\cite{woowoo} jeho typ, který je uvozen tečkou a je následován dvojtečkou. Na typ prostředí dle \cite{woowoo} přímo
navazuje tělo (text) prostředí bez jakýchkoli bílých znaků (mezer; a to před tělem i v něm).

Pro ilustraci viz zobecněný formát definice krátké formy vnitřního prostředí \ref{format-woowoo-inner-env-short} nebo
definice konkrétního typu krátké formy vnitřního prostředí ze studijního textu BI-PKM \ref{format-woowoo-pkm-cite}.

\begin{listing}
    \caption{Obecný formát definice krátké formy vnitřního prostředí WooWoo dokumentu}
    \label{format-woowoo-inner-env-short}
    \begin{minted}[breaklines]{text}
... text text .vnitrniProstredi:tělo text text ...
    \end{minted}
\end{listing}

\begin{listing}
    \caption{Krátká forma vnitřního prostředí ve zdroji studijního textu k BI-PKM \cite{pkm}}
    \label{format-woowoo-pkm-cite}
    \begin{minted}[breaklines]{text}
.cite:ieee754
    \end{minted}
\end{listing}

Verbózní forma vnitřního prostředí má dle \cite{woowoo} postfixovou podobu. Definici takového prostředí tvoří dle
\cite{woowoo} jeho tělo (text) obklopené dvojitými horními uvozovkami. Na tělo prostředí dle \cite{woowoo} přímo
navazuje jeho typ, který je uvozen tečkou (mezi ukončující uvozovkou těla a tečkou není mezera). Tělo prostředí může
v sobě dále obsahovat další vnitřní prostředí. Na typ prostředí může dle \cite{woowoo} dále volitelně navazovat číselná
reference do meta-bloku textového bloku, v němž je vnitřní prostředí obsaženo. Tato reference je od typu prostředí dle
\cite{woowoo} opět oddělena tečkou.

Pro ilustraci viz zobecněný formát definice verbózní formy vnitřního prostředí \ref{format-woowoo-inner-env-verbose}
nebo definice konkrétního typu verbózní formy vnitřního prostředí (s číselnou referencí) ze studijního textu BI-PKM
\ref{format-woowoo-pkm-notion}.

\begin{listing}
    \caption{Obecný formát definice verbózní formy vnitřního prostředí WooWoo dokumentu}
    \label{format-woowoo-inner-env-verbose}
    \begin{minted}[breaklines]{text}
... text text ...
... text "tělo tělo".vnitrniProstredi.1 text ...
... text text ...
  1:
    klic: hodnota
    \end{minted}
\end{listing}

\begin{listing}
    \caption{Verbózní forma vnitřního prostředí ve zdroji studijního textu k BI-PKM \cite{pkm}}
    \label{format-woowoo-pkm-notion}
    \begin{minted}[breaklines]{text}
"čísla s plovoucí desetinnou čárkou".notion.1
    \end{minted}
\end{listing}

Dále dle \cite{woowoo} existují dvě zkrácené formy verbózních vnitřních prostředí, které používají před jeho typem
namísto tečky zavináč \mintinline{text}{@}, respektive křížek \mintinline{text}{#}. Za zavináčem je pak namísto
standardního formátu typu vnitřního prostředí číslo. Tyto zkrácené formy už nemohou být následovány číselnou
referencí. Chování těchto zkrácených forem je dále určeno šablonou. Ve starší verzi WooWoo se navíc namísto zavináče
používala tečka.

\subsection{Další konstrukty}

WooWoo dokument může obsahovat jednořádkové komentáře, které jsou uvozeny procentem \mintinline{text}{%}. Komentáře
nemohou být v těle křehkých vnějších prostředí. V současnosti ve \cite{woowoo} není definován způsob, jak znak procenta
vložit mimo křehká vnější prostředí tak, aby nezačal komentář.

Dokument také může být rozdělen do více souborů. Obsah z jiného souboru se do dokumentu vloží s pomocí direktivy
\mintinline{text}{.include cesta/k/souboru.woo}. Tato direktiva v současnosti není ve specifikaci \cite{woowoo} popsána
a není tak například známo, jak přesně by se vložil soubor s mezerami ve své cestě.

\subsection{FIT Template}

FIT Template je WooWoo šablona, která je aktuálně v praxi používána pro všechny existující studijní texty vytvořené s
pomocí WooWoo.

Specifikace \cite{woowoo} uvádí, že tato šablona dělí dokument na následující části: kapitoly (\mintinline{text}
{.Chapter}), sekce (\mintinline{text}{.Section}) a podsekce (\mintinline{text}{.Subsection}). Části dokumentu mají dle
\cite{woowoo} jednu vyžadovanou položku metadat, značku \mintinline{text}{label}. Tyto značky jsou dle \cite{woowoo}
využívány pro reference částí dokumentu.

\begin{sloppypar}
Bloky tato šablona podle \cite{woowoo} považuje za odstavce textu. FIT Template dle \cite{woowoo} poskytuje následující
objekty: \mintinline{text}{.Definition}, \mintinline{text}{.Corollary}, \mintinline{text}{.Lemma}, \mintinline{text}
{.Theorem}, \mintinline{text}{.Proof}, \mintinline{text}{.Remark}, \mintinline{text}{.Example}, \mintinline{text}
{.Figure}, \mintinline{text}{.Question} a \mintinline{text}{.Table}.
\end{sloppypar}

Dále dle \cite{woowoo} FIT Template poskytuje následující vnější prostředí: \mintinline{text}{.align}, \mintinline{text}
{.caption}, \mintinline{text}{!codeblock}, \mintinline{text}{.enumerate}, \mintinline{text}{.equation}, \mintinline
{text}{.image}, \mintinline{text}{.itemize}, \mintinline{text}{.quote}, \mintinline{text}{!sage}, \mintinline{text}
{.solution}, \mintinline{text}{!tabular} a \mintinline{text}{!tikz}. Vnější prostředí bez hlavičky šablona považuje za
zkratku pro prostředí \mintinline{text}{.equation}.

\begin{sloppypar}
Tato šablona poskytuje dle \cite{woowoo} následující vnitřní prostředí: \mintinline{text}{.cite}, \mintinline{text}
{.code}, \mintinline{text}{.emphasize}, \mintinline{text}{.eqref}, \mintinline{text}{.footnote}, \mintinline{text}
{.item}, \mintinline{text}{.math}, \mintinline{text}{.notion}, \mintinline{text}{.quoted}, \mintinline{text}{.reference}
a \mintinline{text}{.todo}. Zkrácenou formu vnitřního prostředí s mřížkou FIT Template dle \cite{woowoo} považuje za
zkratku \mintinline{text}{.reference}, zkrácenou formu se zavináčem pak za číselý odkaz do meta-bloku. Dále dle
\cite{woowoo} šablona určuje umístění textu mezi dva dolary (\mintinline{text}{$text text$}) jako zkratku pro prostředí
\mintinline{text}{.math}.
\end{sloppypar}

\subsection{Problematická místa}

Vzhledem k tomu, že je WooWoo nový formát, některé věci v něm nejsou ještě zcela dořešené. Například možnost vložení
procenta mimo křehká prostředí bez toho, aby začalo nový řádkový komentář. Stejně tak ve FIT Template šabloně ještě
nejsou zcela dořešena například vnější prostědí \mintinline{text}{.enumerate}, \mintinline{text}{.itemize} či
\mintinline{text}{!tabular}.

S tím jak se formát vyvíjí, obsahují existující dokumenty syntaktické prvky ze starších verzí WooWoo, jako například
zkrácenou formu vnitřního prostědí s tečkou namísto zavináče. Viz příklad \ref{format-woowoo-pkm-old-shorthand} přímo ze
zdroje studijního textu k BI-PKM.

\begin{listing}
    \caption{Starší syntaxe zkrácené formy vnitřního prostředí ve zdroji studijního textu k BI-PKM \cite{pkm}}
    \label{format-woowoo-pkm-old-shorthand}
    \begin{minted}[breaklines]{text}
Matematika je velmi úzce spjata s tzv.
"vědeckou metodou poznání".1, o které lze bez velkého přehánění
prohlásit, že tvoří základ naší civilizace. Častým lidským cílem
je hledání hlubšího porozumnění světa a řešení různých problémů.
Matematika v této činnosti nehraje roli pouhého početní stroje.
Nejprve je nutné problém analyzovat, rozebrat ho na části a
zkoumat jejich vztahy a chování. Typicky dojde k vytvoření
matematického modelu, který (více či méně) dobře vystihuje náš
problém. Následně se v rámci tohoto abstraktního modelu snažíme
dobrat k závěrům a poznání původního problému.
  1:
    link: 'https://en.wikipedia.org/wiki/Scientific_method'
    \end{minted}
\end{listing}

Existující dokumenty také obsahují prvky, které čistě z pohledu specifikace WooWoo formátu nejsou zcela validní.
Například prostředí \mintinline{text}{!tabular} v sobě obsahují vniřní prostředí s matematickými výrazy \cite{pkm}, viz
příklad ze zdroje studijního textu k BI-PKM \ref{format-woowoo-pkm-table}. Pro jejich korektní zpracování by tak muselo
dojít k další transformaci, přestože jde o obsah křehkého prostředí. Obdobně například položka \mintinline{text}{title}
v meta-blocích některých objektů obsahuje matematické výrazy (navíc pro ně ani není použita stejná syntaxe jako pro
zkratku vnitřního matematického prostředí \mintinline{text}{.math}, viz příklad ze zdroje studijního textu k BI-PKM
\ref{format-woowoo-pkm-title}).

\begin{listing}
    \caption{Křehké prostředí \mintinline{text}{!tabular} obsahující matematické výrazy ve zdroji studijního textu k
      BI-PKM~\cite{pkm}}
    \label{format-woowoo-pkm-table}
    \begin{minted}[breaklines]{text}
!tabular:

    Řecké písmeno  Kapitálka  Česká výslovnost LaTeX
    -------------  ---------  ---------------- -------
    $\alpha$                    alfa           alpha
    $\beta$                     beta           beta
    $\gamma$       $\Gamma$     gama           gamma
    $\delta$       $\Delta$     delta          delta
    $\epsilon$                  epsilon        epsilon
    $\zeta$                     zeta           zeta
    $\eta$                      éta            eta
    $\theta$       $\Theta$     théta          theta
    $\kappa$                    kapa           kappa
    $\lambda$      $\Lambda$    lambda         lambda
    \end{minted}
\end{listing}

\begin{listing}
    \caption{Meta-blok obsahující matematické výrazy ve zdroji studijního textu k BI-PKM \cite{pkm}}
    \label{format-woowoo-pkm-title}
    \begin{minted}[breaklines]{text}
.Proof:
  title: 'Důkaz iracionality \(\sqrt{2}\)'
  show: true
    \end{minted}
\end{listing}

Dále existující dokumenty obsahují prvky, které jsou zcela validní, ale jejich zobrazení v náhledu nebude jednoduché.
Jedná se například o číselné reference do meta-bloku ve vnitřních i vnějších prostředích obsahujících matematické
výrazy.

Formát také obsahuje prvky, které nejsou regulární (například správné párování začátků a konců vnitřních prostředí) a
také prvky, které nejsou ani bezkontextové (například změna významu na základě úrovně odsazení je kontextuálně závislá).

Tato problematická místa bude potřeba dále zvážit ať už při tvorbě parseru, zobrazení náhledu, implementaci navigačních
prvků, realizaci zvýrazňování syntaxe nebo výběru knihoven a dalších nástrojů.
