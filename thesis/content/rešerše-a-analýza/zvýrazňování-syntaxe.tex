Podle \cite{atom-docs} v sobě Atom nabízí zabudované dva způsoby, jak přidat podporu zvýrazňování syntaxe pro nové
jazyky.

První, dle \cite{atom-docs} novější způsob je založený na nástroji Tree-sitter. Tree-sitter je dle \cite
{tree-sitter-docs} parser generátor, který na základě bezkontextové gramatiky zapsané v jeho vlastním formátu vygeneruje
parser, jehož výstupem je derivační strom (CST). Aby pak fungoval efektivně, mělo by dle \cite{tree-sitter-docs} navíc
jít o LR(1) gramatiku. Vygenerovaný parser je dle \cite{tree-sitter-docs} v jazyku C a lze ho pak použít i v jiných
aplikacích. Aby mohl být použit v Atomu, musí být dle \cite{atom-docs} zkompilován pro požadované platformy, spolu s tím
vygenerovány \textit{bindings} pro Node.js a tento celý výstup publikován jako knihovna na npm. Tuto knihovnu následně
dle \cite{atom-docs} využije vytvářený \textit{package} přidávající do Atomu podporu nového jazyka s pomocí \textit
{grammar} souboru v CSON nebo JSON formátu, ve kterém dále specifikuje, které vrcholy CST připadají kterému \textit
{grammar scope} (\textit{grammar scopes} v podstatě určují CSS třídu, která bude kódu, který odpovídá danému vrcholu,
přiřazena). Těmto \textit{grammar scopes} pak dle \cite{atom-docs} s pomocí odpovídajících CSS tříd přiřazují styly
různé další \textit{packages} přidávající do Atomu syntaktická témata.

Výhodou tohoto přístupu je, že podle \cite{atom-docs} funguje rychleji a přesněji (parser je založený na bezkontextové
gramatice namísto rozšířených regulárních výrazů), než dále zmíněná alternativa. Nevýhodou je, že i dle Tree-sitter
dokumentace \cite{tree-sitter-docs} je náročný na realizaci. Navíc vzhledem ke kontextově závislým prvkům WooWoo formátu
by takto vzniklý parser stejně nebyl schopen WooWoo parsovat zcela korektně a nemohl by tak být sám o sobě využit
například pro překlad do HTML pro náhled dokumentu.

Druhým způsobem je dle \cite{atom-docs} vytvoření gramatiky ve formátu založeném na TextMate gramatikách. Jedná se o
poměrně jednoduchý systém, který dle \cite{atom-docs} využívá (rozšířené) regulární výrazy. Postup vytvoření \textit
{grammar} souboru je zde dle \cite{atom-docs} podobný, jako v předchozím případě, pouze namísto vrcholů CST přiřazujeme
\textit{grammar scopes} vzorům popsaným (rozšířenými) regulárními výrazy (a samozřejmě zde odpadá definice použité
parser knihovny). Vnitřně pak dle \cite{atom-docs} Atom využívá engine (rozšířených) regulárních výrazů Oniguruma,
konkrétní podporované konstrukty v (rozšířených) regulárních výrazech jsou tedy dané právě tímto enginem.

Výhodou tohoto přístupu je jednoduchost implementace. Další výhodou je, že podobný způsob zvýrazňování syntaxe využívají
další populární editory (např. TextMate, po kterém je tento způsob zvýrazňování syntaxe pojmenovaný, dále třeba VS Code
\cite{vscode-docs} nebo Sublime Text \cite{sublime-text-docs}) a vytvořená gramatika by se tak dala s drobnými úpravami
použít i v nich. Nevýhodou tohoto přístupu je již zmiňovaná nižší rychlost a přesnost. Dále je nutno zmínit, že
Tree-sitter gramatiky jsou aktuálně dle \cite{atom-docs} preferovány a Atom na ně chce postupně přejít. Nicméně vzhledem
k velkému množství existujících TextMate gramatik není pravděpodobné, že by k nějakému úplnému ukončení jejich podpory
došlo v dohledné době.

Atom dále dle \cite{atom-docs} umožňuje namísto JSON/CSON souboru popsat gramatiku přímo v rámci API a dynamicky ji měnit
za běhu. Toho využivá například \textit{package} tasks \cite{atom-package-tasks}. V našem případě by toto mohlo být
užitečné například pro přesnější zvýrazňování syntaxe v závislosti na použité šabloně. Dále by tento přístup mohl
zmenšit opakování kódu (za předpokladu že by obdobné regulární výrazy byly využity také pro parsování) a lepší
testovatelnost.
