Instalace rozšíření vyžaduje Atom 1.54 a novější (starší verze mohou fungovat, není to však otestováno).

Zobrazení TikZ obrázků dále vyžaduje lokální instalaci \TeX{} distribuce, která musí poskytovat příkazy
\mintinline{text}{latex} a \mintinline{text}{dvisvgm}. Dále musí \TeX{} distribuce poskytovat používané \textit
{packages} (v případě použití výchozí preambule dokumentu jde o následující packages: amsmath, amssymb, bbding, fontenc,
inputenc, libertine, pgfplots a standalone). Pokud TikZ obrázky nebudou použity, instalace \TeX{} distribuce není nutná.

\begin{sloppypar}
Rozšíření je možno nainstalovat buď z terminálu s pomocí příkazu \mintinline{text}{apm install wootom}, nebo přímo
prostřednictvím uživatelského prostředí Atomu (File $\to$ Settings $\to$ Install $\to$ Vyhledejte \uv{wootom} $\to$
Install).
\end{sloppypar}

\begin{sloppypar}
Při práci se zdroji existujících studijních textů (BI-PKM, BI-ZMA) je dále výrazně doporučeno upravit nastavení
\textit{package} kvůli podpoře vlastních \LaTeX{} maker (a přidaných \LaTeX{} \textit{packages}). Nastavení
\textit{package} je možno upravit v File $\to$ Settings $\to$ Packages $\to$ Najděte \textit{package} wootom
$\to$ Settings. Do textového pole položky \textit{MathJax Macros} vložte obsah souboru \mintinline{text}
{settings/mathjax-macros.txt} z přiloženého CD. Do textového pole položky \textit{TikZ Preamble} pak obdobně vložte
obsah souboru \mintinline{text}{settings/tikz-preamble.txt}. Po změně nastavení je doporučeno znovu načíst Atom (pro to
může být využito klávesové zkratky \textsc{Ctrl + Shift + F5}).
\end{sloppypar}

\section*{Nastavení pro vývoj}

Pro vývoj rozšíření je navíc k závislostem z úvodu této přílohy vyžadována instalace Node.js (spolu s npm).

\begin{enumerate}[label=\textbf{\arabic*.}, ref=F\arabic*.]
    \item Získejte zdroj rozšíření (GitHub\footnote{GitHub: Wootom (\url{https://github.com/davidstraka2/wootom})},
        přiložené CD, \ldots)
    \item \mintinline{text}{cd} do adresáře se zdrojem
    \item \mintinline{text}{apm link}
    \item \mintinline{text}{npm install}
    \item \mintinline{text}{npm run build}
\end{enumerate}

\subsection*{Užitečné příkazy}

\begin{itemize}
    \item \mintinline{text}{npm test} – Spuštění jednotkových a integračních testů
    \item \mintinline{text}{npm run build} – Kompilace zdroje
    \item \mintinline{text}{npm run format:check} – Kontrola formátování
    \item \mintinline{text}{npm run format:fix} – Oprava formátování
    \item \mintinline{text}{npm run lint:check} – Statická analýza
    \item \mintinline{text}{npm run lint:fix} – Statická analýza s pokusem o nápravu
    \item \mintinline{text}{npm run pack} – Vytvoření produkčního \textit{package}
\end{itemize}
